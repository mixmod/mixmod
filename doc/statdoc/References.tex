\begin{description}
\item[    ] Aitchison, J. and Aitken, C. G. G. (1976), ``Multivariate Binary Discrimination by the Kernel Method,"
{\em Biometrika}, {\em 63}, 413-420.
%\item[    ] Aitkin, M. and Rubin, D. B. (1985), ``Estimation and Hypothesis
%Testing in Finite Mixture Models," {\em Journal of the Royal
%Statistical Society}, Series B, {\em 47}, 67-75.
%\item[    ] Aitkin, M. and Tunnicliffe Wilson, G. (1980), ``Mixture Models,
%Outliers and the EM Algorithm," {\em Technometrics}, {\em 22}, 325-332.
%\item[    ] Akaike, H. (1974), ``A New Look at the Statistical
%Identification Model," {\em IEEE Transactions on Automatic Control}, {\em 19}, 716-723.
\item[    ] Banfield, J. D. and Raftery, A. E. (1993), ``Model-Based
Gaussian and non Gaussian Clustering," {\em Biometrics,} {\em 49}, 803-821.
%\item[    ] Bensmail, H. and Celeux G. (1996), ``Regularized Gaussian discriminant analysis through eigenvalue decomposition''. Journal of the American Statistical Association, {\em 91}, 1743-48.
%\item[    ] Bezdek, J. C. (1981), {\em Pattern Recognition with Fuzzy
%Objective Function Algorithms}, New York: Plenum.
\item[    ] Biernacki, C. Celeux, G. and Govaert, G. (1999), ``An improvement of the NEC criterion for assessing the number of components arising from a mixture,''
{\em Pattern Recognition letters}, No 20, 267-272.
\item[    ] Biernacki, C. and Govaert, G. (1999). Choosing Models in Model-based Clustering and Discriminant Analysis.
{\em Journal of Statistical Computation and Simulation, 64, 49-71}.
\item[    ] Biernacki, C. Celeux, G. and Govaert, G. (2000), ``Assessing a Mixture Model for Clustering with the Integrated Completed Likelihood,"
{\em IEEE Transactions on Pattern Analysis and Machine Intelligence}, vol {\em 22}, No 7, 719-725.
\item[    ] Biernacki, C. Celeux, G. and Govaert, G. (2003) "Choosing starting values for the EM algorithm for getting the highest
likelihood in multivariate Gaussian mixture models''. {\em Computational Statistics and Data Analysis}, {\em 41}, 561-575.
%\item[    ] Bock, H. H. (1985), ``On Tests Concerning the Existence of a
%Classification," {\em Journal of Classification}, {\em 2}, 77-108.
%\item[    ] Bock, H. H. (1989), ``Probabilistic Aspects in Cluster
%Analysis," in {\em Conceptual and Numerical Analysis of Data}, O. Opitz (ed.)
%Springer-Verlag, Heidelberg, pp. 12-44.
\item[    ] Bouveyron C., Girard S. and Schmid C., High Dimensional Discriminant Analysis, {\em Communications in Statistics:
Theory and Methods, 36}, 2607-2623.
%\item[    ] Bozdogan, H. (1990), ``On the Information-Based Measure of
%Covariance Complexity and its Application to the Evaluation of
%Multivariate Linear Models," {\em Communications in Statistics, Theory
%and Methods} {\em 19}, 221-278.
\item[    ] Bozdogan, H. (1993), ``Choosing the Number of Component Clusters in the
Mixture-Model Using a New Informational Complexity Criterion of the
Inverse-Fisher Information Matrix," in {\em Information and Classification},
O. Opitz, B. Lausen, and R.
Klar (eds.), Heidelberg: Springer-Verlag,  pp. 40-54.
%\item[    ] Bozdogan, H. and Sclove, S. L. (1984), ``Multi-Sample Cluster
%Analysis using Akaike 's Information Criterion," {\em Annals of Institute
%of Statistical Mathematics}, {\em 36}, 163-180.
%\item[    ] Bryant, P. G. (1991), ``Large-Sample Results for
%Optimization Based Clustering Methods," {\em Journal of
%Classification}, {\em 8}, 31-44.
%\item[    ] Bryant, P. G. (1993), ``On Detecting the Numbers of Clusters
%Using the MDL Principle,"
%Unpublished Manuscript.
%\item[    ] Bryant, P. G. and Williamson, J. A. (1978), ``Asymptotic Behavior
%of Classification Maximum Likelihood Estimates," {\em Biometrika}, {\em 65}, 273-281.
%\item[    ] Bryant, P. G. and Williamson, J. A. (1986), ``Maximum Likelihood
%and Classification: a Comparison of Three Approaches," in
%{\em Classification as a tool of research}, W. Gaul and M. Schader (eds.)
%North-Holland,   pp. 33-45.
%\item[    ] Celeux G. and Diebolt J. (1985) "The SEM algorithm : a probabilistic teacher algorithm derived from the EM algorithm for the mixture problem". {\em Comp. Statis. Quaterly, 2, 73-82}.
%\item[    ] Celeux, G. (1986), ``Validity Tests in Cluster Analysis
%Using a Probabilistic Teacher Algorithm," {\em COMPSTAT 90}, F. de Antoni, N. Lauro and A. Rizzi
%(eds.)
%Heidelberg: Springer-Verlag, pp. 163-169.
\item[    ] Celeux, G. and Govaert, G. (1991), ``Clustering Criteria for Discrete
Data and Latent Class Models," {\em Journal of Classification}, {\em 8}, 157-176.
\item[    ] Celeux, G. and Govaert, G. (1992). A classification EM
algorithm for clustering and two stochastic versions. {\em
Computational Statistics \& Data Analysis}, {\bf 14}, 315-332.
%\item[    ] Celeux, G. and Govaert, G. (1993), ``Comparison of the Mixture
%and the Classification Maximum Likelihood in Cluster Analysis,"
%{\em Journal of Statistical Computation and Simulation}, {\em 47}, 127-146.
\item[    ]  Celeux, G. and Govaert, G. (1995) "Parsimonious Gaussian models in cluster analysis". {\em Pattern Recognition, 28, 781-793}.
\item[    ] Celeux, G. and Soromenho, G. (1996) "An entropy criterion for assessing the number of clusters in a mixture model". {\em Journal of Classification, 13, 195-212}.
%\item[    ] Culter, A. and Windham, M. P. (1993), ``Information-Based
%Validity Functionals for Mixture Analysis," {\em Proceedings of the
%first US-Japan Conference on the Frontiers of Statistical Modeling.}
%(H. Bozdogan ed.),  Amsterdam: Kluwer, pp. 149-170.
\item[    ] Dempster, A. P., Laird, N. M. and Rubin, D. B. (1977). Maximum likelihood from incomplete data via the EM algorithm (with discussion). {\em J. R. Statis. Soc. B}, {\bf 39}, 1-38.
\item[    ] Everitt, B. (1984). {\em An Introduction to Latent Variable Models}. London,
Chapman and Hall.
\item[    ] Fraley, C. and Raftery, A. E. (1998): How Many Clusters ? Answers
via Model-based Cluster Analysis. {\em The Computer Journal, 41}, 578-588.
215-231.
%\item[    ] Flury, B. W. (1984). Common principal components in $k$ groups. {\em JASA}, {\bf 79}, 892-897.
\item[    ] Flury, B. W., Gautschi, W. (1986). An algorithm for simultaneous orthogonal transformation of several positive
definite symmetric matrices to nearly diagonal form. {\em SIAM J. Scientific Statist. Comput.}, {\bf 7}, 169-184.
%\item[    ] Flury, B. W., Schmid, M. J. and Narayanan, A. (1993) Error rates in quadratic
%discrimination with contraints on the covariance matrices. {\em Journal of Classification} (to appear).
\item[    ] Friedman, H. P. and Rubin, J. (1967). On some invariant
criteria for grouping data. {\em JASA,} {\bf 62}, 1159-1178.
%\item[    ] Ganesalingam, S. (1989), ``Clasification
%and Mixture Approaches to Clustering via Maximum Likelihood," {\em
%Applied Statistics}, {\em 38}, 455-466.
\item[    ] Goodman, L. A. (1974), ``Exploratory Latent Structure Analysis Using Both Identifiable and Unidentifiable Models," {\em Biometrika}, {\em 61},
%\item[    ] Hathaway, R. J. (1986), ``Another Interpretation of the EM
%Algorithm for Mixture Distributions," {\em Statistics and Probability
%Letters}, {\em 4}, 53-56.
\item[    ] Keribin, C. (2000). Consistent estimation of the order of mixture.
{\em Sankhya}, {\em 62}, 49-66.
%\item[    ] Koehler, A. B. and Murphree, E. H. (1988), ``A Comparison of
%the Akaike and Schwarz Criteria for Selecting Model Order," {\em
%Applied Statistics}, {\em 37}, 187-195.
%\item[    ] Macqueen, J. (1967), ``Some Methods for Classification and
%Analysis of Multivariate Observations," {\em Proceedings of the 5th
%Berkeley Symposium on Mathematical Statistics and Probability},
%L. M. Le Cam and J. Neyman (eds.), Berkeley: University of California Press, Vol. 1 pp. 281-297.
\item[    ] Maronna, R. and Jacovkis, P. M. (1974). Multivariate procedures with variable metrics. {\em Biometrics}, {\bf 30}, 499-505.
%\item[    ] Marriott, F. H. C. (1975), ``Separating Mixtures of Normal Distributions,''
%{\em Biometrics, 31}, 767-769.
%\item[    ] Marriott, F. H. C. (1982). Optimization methods of cluster analysis. {\em Biometrika}, {\bf 69}, 239-249.
%\item[    ] McLachlan, G. J. (1987), ``On Bootstraping the Likelihood
%Ratio Test Statistic for the Number of Components in a Normal Mixture,"
%{\em Applied Statistics} {\em 36}, 318-324.
\item[    ] McLachlan, G. J. (1982). The classification and mixture maximum likelihood approaches to cluster analysis.
in {\em Handbook of Statistics} (Vol. 2), P. R. Krishnaiah and L. N. Kanal (Eds.). Amsterdam: North-Holland, pp. 199-208.
%\item[    ] McLachlan, G. J. and Basford K. E. (1989). {\em Mixture
%Models, Inference and Applications to Clustering}. New
%York, Marcel Dekker.
\item[    ] McLachlan, G. J. and Peel D. (2000). {\em Finite Mixture Models}. New York, Wiley.
\item[    ] McNicholas, P.D. and Murphy, T.B. (2008) Parsimonious Gaussian Mixture Models. {\em Statistics and Computing}, to appear.
%\item[    ] Rissanen, J. (1989),  {\em Stochastic Complexity  in
%Statistical Inquiry}, Teaneck,
%New Jersey: World Scientific.
\item[    ] Schwarz, G. (1978), ``Estimating the Dimension of a Model,"
{\em Annals of Statistics}, {\em 6}, 461-464.
\item[    ] Scott, A. J. and Symons, M. J. (1971). Clustering methods
based on likelihood ratio criteria. {\em Biometrics}, {\bf 27}, 387-397.
%\item[    ] Soromendho, G. (1994), ``Comparing Approaches for Testing the
%Number of Components in a Finite Mixture Model," {\em Computational
%Statistics, 9,} 65-78.
%\item[    ] Symons, M. J. (1981). Clustering criteria and multivariate normal mixtures. {\em Biometrics}, {\bf 37}, 35-43.
\item[    ] Tipping, M. E. and C. M. Bishop (1999). Mixtures of probabilistic principal component analysers. N{\em eural Computation  11}, 443-482.
%\item[    ] Titterington, D. M., Smith, A. F., and Markov, U. E. (1985),
%{\em Statistical Analysis of Finite Mixture Distributions}, New York: Wiley.
\item[    ]Ward, J.H. (1963) Hierarchical groupings to optimize an objective function. {\em JASA}, {\bf 58}, 236-244.
%\item[    ] Windham, M. P. and Cutler, A. (1992), ``Information Ratios
%for Validating Cluster Analyses," {\em Journal of the American
%Statistical Association}, {\em 87}, 1188-1192.
%\item[    ] Wolfe, J. H. (1970), ``Pattern Clustering by Multivariate
%Mixture Analysis," {\em Multivariate Behavioral Research}, {\em 5}, 329-350.
%\item[    ] Wolfe, J. H. (1971), ``A Monte Carlo Study of the Sampling Distribution
%of the Likelihood Ratio for Mixtures of Multinormal
%Distributions," US Naval Personnel Research Activity.
%{\em Technical Bulletin} {\em STB 72-2}, San Diego, California.
\end{description}

%%%%%%%%%%%%%%%%%%%%%%%%%%%%%%%%%%%%%%%%%%%%%%%%%%%%