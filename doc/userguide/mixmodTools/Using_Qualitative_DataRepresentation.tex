%====================================
%====================================
\subsection{Individuals representation}
%====================================
%====================================
{\sc mixmod} is concerned with data set where experimental unit
individuals are described with several variables and each
variables with several modalities. Individuals are represented in
a standard way: a row vector. Then data sets are represented by a
matrix with $n$ rows and $d$ columns, rows representing individuals, and columns representing variables.\\
In data file, modalities are represented by integers: 1 represent
the first modality of considered variable. Binary data is not
represented in the standard way (by 0 and 1), 0
is replaced by 1 and 1 is replaced by 2 (modalities 1 and 2 of considered variable).\\
Let consider the b\_toby.dat individual file, in the directory
MIXMOD/DATA, which describes 216 individuals, 4 variables:
{\scriptsize
\begin{verbatim}
        2    2    2    2
        2    2    2    2
        2    2    2    2
        2    2    2    2
        ...
\end{verbatim}}


%==================================
%==================================
\subsection{Weight representation}
%==================================
%==================================
Weights are represented in the same way as in quantitative data
sets case.

%=====================================
%=====================================
\subsection{Parameter representation}
%=====================================
%=====================================
Parameters give proportions, centers and scatters of each mixture
component. This representation is used in the cases of USER and USER\_PARTITION initialization types.\\
Let consider the parameter file b\_toby.init with 2 clusters, in
the directory MIXMOD/DATA/. It gives values of the unknown mixture
model parameters. {\scriptsize
\begin{verbatim}
        0.720754                                 [Initial proportion of component 1]
        2  1  1  1                               [Initial center of component 1]
        0.286412  0.329619  0.354016  0.132373   [Initial scatter matrix of component 1]

        0.279246                                 [Initial proportion of component 2]
        2  2  2  2                               [Initial center of component 2]
        0.006807  0.060236  0.073469  0.230868   [Initial scatter matrix of component 2]
\end{verbatim}}


%==============================================
%==============================================
\subsection{Partition representation}
%==============================================
%==============================================
Partition are represented in the same way as in quantitative data
sets case.
