%====================================
%====================================
\subsection{Individuals representation} \label{individualsRepresentationSection}
%====================================
%====================================
{\sc mixmod} is concerned with data sets where experimental unit
individuals are described with several variables. Individuals are
represented in a standard way a row vector. Then data sets are
represented by a matrix with $n$ rows and $d$ columns, rows representing individuals, and columns representing variables.\\
Let's consider the {\it esteve.dat} individual file, in the directory MIXMOD/DATA, which describes 77 individuals and 7
variables in a \textbf{quantitative situation}

{\scriptsize
\begin{verbatim}
        0.8277   0.3626   0.4243   0.0269   0.0523   0.0000   2.7021
        0.7713   0.4399   0.4586   0.0355   0.0000   0.0015   4.0874
        0.6407   0.4841   0.5302   0.0000   0.2723   0.0000   2.7531
        0.9384   0.2396   0.2436   0.0008   0.0516   0.0000   2.6013

        ...
\end{verbatim}}


Let's consider the {\it b\_conf.dat} individual file, in the directory MIXMOD/DATA, which describes 30 individuals and 8
variables in a \textbf{qualitative situation}

{\scriptsize
 \begin{verbatim}
        1        1        5        1        1        1        1        2
        1        5        1        1        1        4        2        2
        2        2        5        1        1        3        1        1
        1        5        2        1        1        3        2        1

        ...
 \end{verbatim}
}


Notice that the possible values of qualitative data depend on the number of modalities of the variable.
Only integers are authorized and values such as \textit{yes/no} are not allowed.

%==================================
%==================================
\subsection{Weight representation}
%==================================
%==================================
Weights are stored in a vector of real or integer numbers, with $n$ (number of individuals) rows.
Weight is an option of {\sc mixmod} and for example it can represent the repetition of individuals.
Let's consider the weight file {\it geyser.wgt} in the directory MIXMOD/DATA/. It gives the information of all
individuals of {\it geyser.dat} repeated 2 times.\\

%{\noindent {\em Warning CV criterion gives pertinent results only if weights are integer numbers.}}

%=====================================
%=====================================
\subsection{Parameter representation} \label{parameterRepresentationSection}
%=====================================
%=====================================
%Parameters give
%proportions, means and dispersion or proportions, means, subDimensions, parameters Akj,
%parameters Bk and orientation of each mixture
%component.
This representation is used in the case of USER initialization type.\\
\begin{itemize}
 \item Gaussian

$ $

Let's consider the parameter file {\it parameter.init} with 2
clusters, in the directory MIXMOD/DATA/. It gives the values of the unknown mixture model parameters in the quantitative case
{\scriptsize
\begin{verbatim}
        0.367647                [Initial proportion of component 1]
        2.094330   54.750000    [Initial mean of component 1]
        17.280889   0.000000    [Initial variance matrix of component 1]
        0.000000   17.280889


        0.632353                [Initial proportion of component 2]
        4.297930   80.284884    [Initial mean of component 2]
        15.830206   0.000000    [Initial variance matrix of component 2]
        0.000000   15.830206
\end{verbatim}}

\item Multinomial

$ $

Let's consider the parameter file {\it parameter\_qualitatif.init} with 2 clusters in the directory MIXMOD/DATA/.
It gives the values of the unknown mixture model parameters in the qualitative case (with modalities [2;3;4;5])
{\scriptsize
 \begin{verbatim}
        0.2                     [Initial proportion of component 1]
        1  2  3  4              [Initial mean of component 1]
        0.1     0.1             [Initial dispersion array of component 1]
        0.1     0.2     0.1
        0.1     0.1     0.3     0.1
        0.1     0.1     0.1     0.4   0.1


        0.8                     [Initial proportion of component 2]
        2  3  4  5              [Initial mean of component 2]
        0.5     0.5             [Initial dispersion array of component 2]
        0.25    0.25    0.5
        0.1667  0.1667  0.1667  0.5
        0.125   0.125   0.125   0.125   0.5


 \end{verbatim}
}

\item Gaussian High Dimensional

$ $

Let's consider the parameter file {\it parameter\_HD.init} with 2 clusters in the directory
MIXMOD/DATA. It gives the values of the unknown mixture parameters in the gaussian high dimensional (HD)
case
{\scriptsize
\begin{verbatim}
        0.75                                            [Initial proportion of component 1]
        14.842      11.718     32.014    36.81   13.35  [Initial mean of component 1]
        3                                               [Initial subDimension of component 1]
        144.744604  0.214614   0.101925                 [Initial parameter Akj of component 1]
        0.063887                                        [Initial parameter Bk of component 1]
       -0.262423    0.355093   0.004478                 [Initial parameter Bk oforientation array of component 1]
       -0.170051   -0.888586  -0.207320
       -0.601104    0.057918   0.532327
       -0.687179   -0.071419  -0.105660
       -0.262061    0.275444  -0.813918


        0.25                                            [Initial proportion of component 2]
        13.27       12.138     28.102    32.624  11.816 [Initial mean of component 2]
        3                                               [Initial subDimension of component 2]
        99.333875   0.155693   0.138530                 [Initial parameter Akj of component 2]
        0.049261                                        [Initial parameter Bk of component 2]
       -0.259855    0.127732  -0.473221                 [Initial parameter Bk oforientation array of component 2]
       -0.239538    0.555917   0.750006
       -0.587639   -0.078075  -0.049268
       -0.675526    0.144512  -0.205855
       -0.271003   -0.804774   0.410791
\end{verbatim}
}


\end{itemize}






%==============================================
%==============================================
\subsection{Partition representation} \label{partitionRepresentationSection}
%==============================================
%==============================================
A partition gives a classification of the individuals they are affected to a mixture component.\\
It is a matrix (of 0 and 1)
with $n$ rows and $k$ columns, each row corresponding to an individual and each column indicating the group
membership ($0$ if the individual does not belong to the group associated to this column
and $1$ otherwise).\\
Some individuals can have no group assignment they are represented by a
row of $0$.\\
Let's consider the full partition file {\it geyser3clusters.part} in the directory MIXMOD/DATA/. This file gives a
full classification of {\it geyser.dat} for 3 mixture components.

This representation is used in the case of USER\_PARTITION initialization type and/or with known partition.
{\scriptsize
\begin{verbatim}
        0   1   0    [individual 1 in component 2]
        0   0   1    [individual 2 in component 3]
        0   1   0    [individual 3 in component 2]
        0   0   1    [individual 4 in component 3]
        1   0   0    [individual 5 in component 1]

        ...

        0   1   0    [individual 268 in component 2]
        0   0   1    [individual 269 in component 3]
        1   0   0    [individual 270 in component 1]
        0   0   1    [individual 271 in component 3]
        1   0   0    [individual 272 in component 1]
\end{verbatim}}
%{\noindent Let's consider the partial partition file {\it iris.partial.part}, in the directory
%MIXMOD/DATA/, with 3 clusters.}
%{\scriptsize
%\begin{verbatim}
%        1  0  0    [individual 3 in component 1]
%
%        ...
%
%        0  1  0    [individual 60 in component 2]
%
%        ...
%
%        0  0  1    [individual 115 in component 3]
%
%        ...
%
%        0  0  0    [individual 149: unknown partition]
%        0  0  0    [individual 150: unknown partition]
%\end{verbatim}}
