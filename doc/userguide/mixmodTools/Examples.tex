The data sets have been chosen to illustrate the features of the
methods implemented in {\sc mixmod}. All of them are available in {\it
MIXMOD/DATA/} directory, which contains the following data sets:
%--------------------------------
\subsection*{Old Faithful Geyser}
%--------------------------------

The file {\it geyser.dat} contains 272 observations from the Old
Faithful Geyser in the Yellowstone National Park. Each observation
consists of two measurements: the duration (in minutes) of the eruption
and the waiting time (in minutes) to the next eruption.
Old Faithful erupts more frequently than any other big geyser,
although it is not the largest nor the most regular geyser in the park. Its
average interval between two eruptions is about 76 minutes, varying from
45 - 110 minutes. An eruption lasts from 1.1/2 to 5 minutes, expels 3,700 -
8,400 gallons (14,000 - 32,000 liters) of boiling water, and reaches
heights of 106 - 184 feet (30 - 55m). It was named for its consistent
performance by members of the Washburn Expedition in 1870. Old
Faithful is still as spectacular and predictable as it was a century
ago.

%-------------------------------------
\subsection*{Diabetes in Pima Indians}
%-------------------------------------
The diagnostic, binary-valued variable investigated is whether the
patient shows signs of diabetes according to the World Health Organization
criterion (i.e.\ if the 2 hour post-load plasma glucose was at least
200 mg/dl at any survey examination or if found during routine medical
care). The population of women who were at least 21 years old, lives
near Phoenix, Arizona, USA.\@

It appears that this data set was incorrectly recorded. Many
attributes have missing values and these have been encoded with the
numerical value 0. This applies to something like 5 of the eight
attributes. Besides, the classification variable is coded as 1=healthy
and 2=diabetic (i.e.\ interchanged wrt. UCI dataset).
\begin{table}[!h]
    \caption{Diabetes in Pima Indians}
    \begin{center}
        \begin{tabular}{|c|c|l|}
            \hline
            Column & Type   & Description \\
            \hline
            1 & metric & number of pregnancies \\
            2 & metric & plasma glucose concentration in an oral glucose tolerance test \\
            3 & metric & diastolic blood pressure (mm Hg) \\
            4 & metric & triceps skin fold thickness (mm) \\
            5 & metric & serum insulin ($\mu$ U/ml) \\
            6 & metric & body mass index (weight in kg/(height in m)$^2$) \\
            7 & metric & diabetes pedigree function \\
            8 & metric & age in years \\
            \hline
        \end{tabular}
    \end{center}
\end{table}

%----------------------------
\subsection*{Haemophilia}
%----------------------------
The Haemophilia data set {\it haemophilia.dat}, analyzed by Basford
and McLachlan (1985), consists of two bivariate components,
and its analysis illustrates the caution needed to be exercised in
practice when fitting mixture models.


%------------------------
\subsection*{Enzyme}
%------------------------

The data set {\it enzyme.dat} concerns the distribution of enzymatic
activity in the blood, for an enzyme involved in the metabolism of
carcinogenic substances, among of group of 245 unrelated
individuals. The interest here is to identify subgroups of slow or
fast metabolisers as a marker of genetic polymorphism in the general
population.  This data set has been analyzed by Bechtel {\em et al.}
(1993), who identified a mixture of 2 skewed distributions using
maximum likelihood technics implemented in the program {\sc Skumix}
of Maclean {\em et al} (1976).

%--------------------
\subsection*{Acidity}
%--------------------
The data set {\it acidity.dat} concerns an acidity index measured in a
sample of 155 lakes in the Northeastern United States and has been
previously analyzed as a mixture of Gaussian distributions on the log
scale by Crawford {\em et al.}(1992, 1994).

%---------------------------
\subsection*{U.S. Companies}
%---------------------------
The data file {\it uscomp.dat} contains measurements for 79
U.S. companies out of the top 500.
\begin{table}[h]
    \caption{U.S. Companies}
    \begin{center}
        \begin{tabular}{|c|c|l|}
            \hline
            Column & Type   & Description \\
            \hline
            1 & text   & company \\
            2 & metric & assets \\
            3 & metric & sales \\
            4 & metric & market value \\
            5 & metric & profits \\
            6 & metric & cash flow \\
            7 & metric & employees \\
            8 & text   & branch \\
            \hline
        \end{tabular}
    \end{center}
\end{table}

%-----------------------------
\subsection*{Swiss Bank Notes}
%-----------------------------
The data file {\it bank.dat} contains measurements on 100 genuine and
100 forged old Swiss 1000 franc bills from Flury and Riedwyl
(1988). The first 100 rows correspond to the genuine bills and the
second 100 rows to the forged data.

\begin{table}[!h]
    \caption{Swiss Bank Notes}
    \begin{center}
        \begin{tabular}{|c|c|l|}
            \hline
            Column & Type   & Description \\
            \hline
            1 & metric & length of the bill\\
            2 & metric & height of the bill, measured on the left \\
            3 & metric & height of the bill, measured on the right \\
            4 & metric & distance of inner frame to the lower border \\
            5 & metric & distance of inner frame to the upper border \\
            6 & metric & length of the diagonal\\
            \hline
        \end{tabular}
    \end{center}
\end{table}
