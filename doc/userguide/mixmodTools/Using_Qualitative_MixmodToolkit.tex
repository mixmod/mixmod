%==================================
\subsection{{\sc mixmod} functions}
%==================================
The using is the same that in quantitative data sets case.\\ In
optional input a vector of number of modalities ($d$ number of
modalities) must be added. The binary optional models must be
chosen: Binary\_p\_E, Binary\_p\_Ek, Binary\_p\_Ej, Binary\_p\_Ekj.\\

%----------------------
\noindent{Example:\\}
%----------------------

\begin{tabular}{c|c}
\begin{minipage}[c]{0.47\columnwidth}%
{\scriptsize
\begin{verbatim}
    -> toby = read('DATA/b_toby.dat',216,4);
    -> nbModalities = [2 2 2 2];
    -> nbCluster = [2 3 4];
    -> criteria = ['BIC' 'ICL'];
    -> models = ['Binay_p_E';
                 'Binay_p_Ek';
                 'Binay_pk_Ej';
                 'Binay_pk_Ekj'];
    -> out = mixmod(toby, nbCluster, nbModalities,...
                    models, criteria);
\end{verbatim}}
\end{minipage}%
&
\begin{minipage}[c]{0.53\columnwidth}%
{\scriptsize
\begin{verbatim}
    >> geyser = load('DATA/b_toby.dat');
    >> nbModalities = [2 2 2 2];
    >> nbCluster = [2 3 4];
    >> criteria = [{'BIC'} {'ICL'}];
    >> models = [{'Binay_p_E'};
                 {'Binay_p_Ek'};
                 {'Binay_pk_Ej'};
                 {'Binay_pk_Ekj'}];
    >> out = mixmod(toby, nbCluster, nbModalities,...
                    models, criteria);
\end{verbatim}}
\end{minipage}%
\end{tabular}


%======================================
\subsection{{\sc mixmodView} functions} \label{mixmodViewFunctionsSection}
%======================================
mixmodView functions are not available for qualitative data sets.


%==================================
\subsection{Others functions}
%==================================

%==========================================
\subsubsection{{\sc printMixmod} functions}
%==========================================
The using is the same that in quantitative data sets case.\\

%----------------------
\noindent{Example:\\}
%----------------------

\begin{tabular}{c|c}
\begin{minipage}[c]{0.47\columnwidth}%
{\scriptsize
\begin{verbatim}
    -> exec('initMixmod.sci');
    -> toby   = read('DATAb\b_toby.dat',216,4);
    -> output = mixmod(toby,2,[2 2 2 2]);
    -> printMixmod(output);
\end{verbatim}}
\end{minipage}%
&
\begin{minipage}[c]{0.53\columnwidth}%
{\scriptsize
\begin{verbatim}
    >> toby   = load('DATA\b_toby.dat');
    >> output = mixmod(toby,2,[2 2 2 2]);
    >> printMixmod(output);
\end{verbatim}}
\end{minipage}%
\end{tabular}\\

%==========================================
\subsubsection{{\sc mixmodInput} functions}
%==========================================
The using is the same that in quantitative data sets case. The
optional input {\em  allBinaryModel} returns, all binary model
types. The optional input {\em allModel} returns all gaussian and
binary model types.\\

%----------------------
\noindent{Example:\\}
%----------------------

\begin{tabular}{c|c}
\begin{minipage}[c]{0.47\columnwidth}%
{\scriptsize
\begin{verbatim}
  -> exec('initMixmod.sci');
  -> toby   = read('DATAb\b_toby.dat',216,4);
\end{verbatim}}
\end{minipage}%
&
\begin{minipage}[c]{0.53\columnwidth}%
{\scriptsize
\begin{verbatim}
  >> toby   = load('DATA\b_toby.dat');
\end{verbatim}}
\end{minipage}%
\end{tabular}

{\scriptsize
\begin{verbatim}
       -> (>>) [criterion, model, strategy] = mixmodInput('allBinaryModel');
       -> (>>) output = mixmod(toby, 2, [2 2 2 2], model);
\end{verbatim}}
