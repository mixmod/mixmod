%=============================
%=============================
\section{System requirements}
%=============================
%=============================
{\sc mixmod} can be used on many systems (unix, linux and windows).\\
%The gcc compiler is required for unix and linux systems. For windows, no compiler is required (a compiled package is available).


%===============================
%===============================
\section{Download and Installation instructions}
%===============================
\hspace{-1cm}
\begin{tabular}{|c|c|l|}

 \hline
Platform & Version & Procedure \\

\hline
%\multicolumn{2}{|r|}{{\bf 3D}} \\

% \cline{1-2}


 & & 1) download mixmod\_2\_1\_1\_linux\_bin.tgz \\
 & binary& 2) put it where you want to install MIXMOD  \\
 & & 3) extract  mixmod\_2\_1\_1\_linux\_bin.tgz \\




{\bf Linux} & & \\
%& \multicolumn{2}{|c|}{} \\
\cline{2-3}
%\hline
&  &  1) download mixmod\_2\_1\_1\_linux\_src.tgz \\
 & & 2) put it where you want to install MIXMOD  \\
 &source & 3) extract  mixmod\_2\_1\_1\_linux\_src.tgz \\
& & 4) launch successively in {\em $<mixmodDir>$} {\em cmake .}, {\em make}, {\em make install}\\
\hline

{\bf Windows} & mixmod\_2\_1\_1.exe & 1) download mixmod\_2\_1\_1.exe\\
& & 2) launch MIXMOD setup and choose {\em $<mixmodDir>$}\\
\hline
\end{tabular}


%===============================
%===============================
%\section{InstallationDownload instructions}
%===============================

%===============================
%===============================
\section{Using {\sc MIXMOD}}
%===============================
\subsection*{Scilab}
Run Scilab and execute the {\em initMixmod.sci} file in directory MIXMOD/ by typing the following command:
\begin{verbatim}
        exec(`initMixmod.sci');
\end{verbatim}
You will be asked to enter MIXMOD directory location, this will load {\em MIXMOD} functions each time you run Scilab.\\
This procedure has to be done only once.
Each time you install a new version of MIXMOD, you will have to launch {\em initMixmod.sci}. To know which
version you are working with, you can type 'mixmodDir' in Scilab consol.




\subsection*{Matlab}

{\sc mixmod} runs with Matlab 7 or above.
Before using {\em MIXMOD}, you have to set {\em mixmod path} in Matlab
by selecting \textit{Set Path} in \textit{File} menu of Matlab window, clic on \textit{Add Folder} then
select {\em MIXMOD} and {\em MIXMOD/UTIL/MATLAB}.
Either you have administrator
rights and thus you can save the new path or you will have to set {\em mixmod path} each time you
want to run {\em mixmod}.
%Launch Matlab and execute {\em 'initMixmod.m'}, mixmod toolkit functions will be added to the path.
%Add {\sc mixmod} to Matlab path (for example, by selecting Set Path in file menu of Matlab window) and run Matlab.





%===============================
%{\sc mixmod} can be obtained via the web for several
%systems, Unix, PC Linux and PC Windows at
%\begin{verbatim}
%        \url{http://www-math.univ-fcomte.fr/mixmod/index.php}
%\end{verbatim}
%{\em mixmod2.1.1-Setup.exe} is an installation version for Windows and {\em
%mixmod2.1.1-Linux-Unix.tgz} (Linux, Unix) is a packed version. They contain all
%the necessary files to use and incorporate {\sc mixmod} in Scilab and Matlab.

%==============================================
%\subsection{Instructions for PC Linux/Unix}
%==============================================
%\begin{itemize}
%\item [1.]
%Download {\em mixmod2.1.1-Linux-Unix.tgz} file on your local account.
%\item [2.]
%The commands to unpack it (and remove the tar file) are
%\begin{verbatim}
%  tar zxvf mixmod2.1.1-Linux-Unix.tgz
%%  rm mixmod2.1.1-Linux-Unix.tgz
%\end{verbatim}
%%\end{itemize}

%=======================================
%\subsection{Instructions for PC Windows}
%=======================================
%\begin{itemize}
%\item [1.]
%Download {\em mixmod2.1.1-Setup.exe} file on your computer.
%\item [2.]
%Launch {\em mixmod2.1.1-Setup.exe}.
%\end{itemize}


%=================================
%=================================
%\section{Make files - Compilation}
%=================================
%=================================

%==============================================
%\subsection{Instructions for PC Linux/Unix}
%==============================================
%This distribution has been tested on
%PC linux. You will need to
%compile all the *.cpp files listed as program files to get the
%complete package. A C/C++ compiler with a make command (which must be compatible with gnumake
%(example gmake)) is needed.

%\begin{itemize}
%  \item [1.]
%    Enter the directory MIXMOD/ to build the package.
%	\begin{verbatim}
%	     cd MIXMOD/
%	\end{verbatim}

%  \item [2.]
%    Configure your system by using the following command
%   	\begin{verbatim}
%	     ./configure
%	\end{verbatim}
%    This will create and/or modify a few makefiles and other files of your
%    system.
%  \item [3.]
%    To compile everything with the CC or gnu compiler use the command
%   	\begin{verbatim}
%	     make
%	\end{verbatim}
%    This will compile {\sc mixmod} and produce the
%    executable files {\em mixmod} and {\em test}.
%  \item [4.]
%    Finally, to complete the building of the package (to update MIXMOD/BIN/ directory), use the command
%	\begin{verbatim}
%	     make install
%	\end{verbatim}
%\end{itemize}

%=======================================
%\subsection{Instructions for PC Windows}
%=======================================
%{\em mixmod2.0-Setup.exe} is an auto-installation file for Windows including an executable file. So, no compilation is needed to install and launch {\sc mixmod} for Windows.\\
%However, to have an executable file well adapted to your computer, you have to compile {\sc mixmod}. In that aim, it is necessary
%to install a {\em gcc} compiler and GNU tools for Windows (example mingw). Then follow the same process as for PC
%Linux/Unix configure, make, make install can be used.


%===============================================
%===============================================
%\section{Environment variables}
%===============================================
%===============================================

%==============================
%\subsection{PC Linux/Unix}
%==============================
%\begin{itemize}
%\item Add the environment variable named {\em PathToMixmod}, which contains {\sc mixmod} directory path.
%\item Add {\sc mixmod} executable path to the environment variable {\em PATH}.
%\end{itemize}

%Example for tcsh or csh shell
%\begin{itemize}
%\item \texttt{setenv PathToMixmod \$HOME/MIXMOD}
%\item \texttt{setenv PATH \$PathToMixmod/BIN:\$PATH}
%\end{itemize}

%Example for sh or bash shell
%\begin{itemize}
%\item \texttt{export PathToMixmod=\$HOME/MIXMOD}
%\item \texttt{export PATH=\$PathToMixmod/BIN:\$PATH}
%\end{itemize}



%============================
%\subsection{PC Windows}
%============================
%If {\sc mixmod} has been installed using the setup file, environment variables have been updated.\\
%Else, the environment variable {\em PathToMixmod} must be
%defined with the {\sc mixmod} directory path. Then add the {\sc mixmod} executable path ({\em \%PathToMixmod\%/BIN}) to the
%home path.\\
%Access to the environment variables
%\begin{verbatim}
%        Control Panel --> System --> Advanced --> Environment Variables
%\end{verbatim}
%For Windows98, environment variables must be updated manually (in {\em autoexe.bat} file for instance).





%================
%================
%\section{Tests}
%================
%================
%The package contains a test program to be sure that {\sc mixmod} is well installed.
%To test the package installation, run the following command under MIXMOD/BIN/
%directory\\
%\begin{itemize}
%\item In a command DOS window for PC Windows
%\begin{verbatim}
 %       test.exe      (to launch a simple test)
%\end{verbatim}
%\item In a shell window for Unix or linux
%\begin{verbatim}
%        ./test     (to launch a simple test)
%\end{verbatim}
%\end{itemize}
%A message diagnostics the test execution {\em failed}, {\em problem(s)} or {\em ok}.







%=================================================
%=================================================
%\section{Using {\sc mixmod} in Scilab environment}
%=================================================
%=================================================
%{\sc mixmod} runs with Scilab-4.0 or above.
%Run Scilab and execute the {\em initMixmod.sci} file in directory MIXMOD/ by typing the following command
%\begin{verbatim}
%        exec(`initMixmod.sci');
%\end{verbatim}


%=================================================
%=================================================
%\section{Using {\sc mixmod} in Matlab environment}
%=================================================
%=================================================
%{\sc mixmod} runs with Matlab 6 or above.
%Add {\sc mixmod} to Matlab path (for example, by selecting Set Path in file menu of Matlab window) and run Matlab.
%\begin{figure}[!ht]
% \epsfxsize=7cm
%  \epsfysize=5cm
%  \centerline{\epsffile{matlabSetPath.ps}}
%  \caption{Set path option.}
%  \label{SetPathMenu}
%\end{figure}

%\begin{figure}[!ht]
%  \epsfxsize=7cm
%  \epsfysize=6cm
%  \centerline{\epsffile{matlabPath.ps}}
%  \caption{Matlab path window.}
%  \label{PathWindow}
%\end{figure}
