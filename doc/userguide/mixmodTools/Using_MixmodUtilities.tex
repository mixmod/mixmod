%================================================
\subsection{ Convert old input file}
%================================================
In the package directory MIXMOD/BIN/ there is an executable file named {\em ConvertOldInputFile}.
 This program convert old input files to new input data files (with keywords since {\sc mixmod} V1.3).\\
\noindent{\bf Synopsis}
{\scriptsize
\begin{verbatim}
       convertOldInputFile nameFile
\end{verbatim}}

{\noindent The conversion creates a new file .old ({\em nameFile.old}) which is the old input data file and the
origin file is replaced by the new input data file with keywords.}\\
Let consider the old input data file geyserOld.xem (which will become geyserOld.xem.old) in the
directory MIXMOD/EXAMPLES/DOC/:
{\scriptsize
\begin{verbatim}
        272                  // NbLines
        2                    // PbDimension
        1                    // NbCriterion
        ICL                  // ListCriterion
        1                    // NbNbCluster
        2                    // ListNbCluster
        1                    // NbModel
        Gaussian_pk_L_Ck     // ListModel
        1                    // NbStrategy
        RANDOM               // InitType
        1                    // NbAlgorithm
        EM                   // Algorithm
        NBITERATION          // StopRule
        200                  // StopRuleValue
        ../DATA/geyser.data  // data file

\end{verbatim}}

\noindent{Command for this example in directory MIXMOD/BIN/:}
{\scriptsize
\begin{verbatim}
      ConvertOldInputFile ../EXAMPLES/DOC/geyserOld.xem

\end{verbatim}}

{\noindent The new geyserOld.xem become:}
{\scriptsize
\begin{verbatim}
        NbLines
            272                 // Size lines
        PbDimension
            2                   // Dimension sample
        NbCriterion
            1                   // Number of criteria
        ListCriterion
            ICL                 // ICL criterion
        NbNbCluster
            1                   // List size of clusters (2)
        ListNbCluster
            2                   // 2 clusters
        NbModel
            1                   // Number of models
        ListModel
            Gaussian_pk_Lk_Ck   // Gaussian ellipsoidal model pk[LkCk]
        NbStrategy
            1                   // One strategy
        InitType
            RANDOM              // Starting parameter by random centers
        NbAlgorithm
            1                   // One algorithm used
        Algorithm
            EM                  // Algorithm EM
        StopRule
            NBITERATION         // Stopping rules for EM, number of iterations
        StopRuleValue
            200                 // 200 iterations desired
        DataFile
            ../DATA/geyser.dat  // Input data set (Old Faithful Geyser)



\end{verbatim}}

%=============================================================
\subsection{ Convert label file to partition file}
%=============================================================
In the package directory MIXMOD/BIN/ there is an executable file named {\em convertLabelFileToPartitionFile}. This program convert a label file, which contains a column vector of label, to a
partition file, which contains a partition matrix of 0 and 1 (between {\sc mixmod} 1.5 and 1.6).\\
\noindent{\bf Synopsis}
{\scriptsize
\begin{verbatim}
       convertLabelFileToPartitionFile nameFile n k
\end{verbatim}}

\noindent{$n$: number of rows (number of samples)}\\
$k$: number of clusters\\
The conversion creates a new partition file .lab ({\em nameFile.lab}) which contains a partition matrix of 0
and 1.\\
Consider the label file geyserLabel.txt in the directory MIXMOD/EXAMPLES/DOC/:
{\scriptsize
\begin{verbatim}
        2    // sample 1 in component 2
        1    // sample 2 in component 1
        2    // sample 3 in component 2
        1    // sample 4 in component 1
        2    // sample 5 in component 2
        1    // sample 6 in component 1
        2    // sample 7 in component 2
        2    // sample 8 in component 2
        1    // sample 9 in component 1
        2    // sample 10 in component 2

        ...
\end{verbatim}}

\noindent{Command for this example MIXMOD/BIN/:}
{\scriptsize
\begin{verbatim}
      convertLabelFileToPartitionFile ../../EXAMPLES/DOC/geyserLabel.txt 272 2
\end{verbatim}}

\noindent{Created file is {\em geyserLabel.txt.lab}:}
{\scriptsize
\begin{verbatim}
        0   1
        1   0
        0   1
        1   0
        0   1
        1   0
        0   1
        0   1
        1   0
        0   1

        ...
\end{verbatim}}



%================================================================
\subsection{ Convert old data weight file in 2 files}
%================================================================
In the package directory MIXMOD/BIN/ there is an executable file named {\em
convertOldWeightDataFileIn2Files}. This program separate old data file with weight in 2 files: a data file and a
weight file (betwenn {\sc mixmod} 1.5 and 1.6).\\
\noindent{\bf Synopsis}
{\scriptsize
\begin{verbatim}
       convertOldWeightDataFileIn2Files nameFile n d
\end{verbatim}}

\noindent{$n$ number of rows (number of samples)}\\
$d$: data dimension (number of columns in data set)\\
The conversion creates two new files. A file .dat ({\em nameFile.dat}) which contains matrix of data set and a
file .wgt ({\em nameFile.wgt}) which contains vector of weight.\\
Consider the old data weight file esteveWeightOld.dat in the directory MIXMOD/EXAMPLES/DOC/:
{\scriptsize
\begin{verbatim}

        0.8277    0.3626    0.4243    0.0269    0.0523    0.        2.7021    10.   [weight = 10]
        0.7713    0.4399    0.4586    0.0355    0.        0.0015    4.0874    10.   [weight = 10]
        0.6407    0.4841    0.5302    0.        0.2723    0.        2.7531    10.   [weight = 10]
        0.9384    0.2396    0.2436    0.0008    0.0516    0.        2.6013    10.   [weight = 10]
        0.7166    0.6576    0.2323    0.        0.007     0.0012    1.9698    10.   [weight = 10]
        0.7802    0.4377    0.4425    0.0595    0.0198    0.0003    2.3369    10.   [weight = 10]
        0.7671    0.3738    0.4918    0.        0.1734    0.0052    3.0098    10.   [weight = 10]
        0.7659    0.4967    0.3769    0.        0.1572    0.        1.7791    10.   [weight = 10]
        0.7448    0.3449    0.5504    0.        0.1526    0.        1.9758    10.   [weight = 10]
        0.7696    0.3979    0.4991    0.0149    0.        0.        3.3411    10.   [weight = 10]

        ...
\end{verbatim}}

\noindent{Command for this example MIXMOD/BIN/:}
{\scriptsize
\begin{verbatim}
      convertOldWeightDataFileIn2Files ../../EXAMPLES/DOC/esteveWeightOld.dat 77 7
\end{verbatim}}

\noindent{First file created is the new data file esteveWeightOld\_noWeight.dat:}
{\scriptsize
\begin{verbatim}
        0.8277    0.3626    0.4243    0.0269    0.0523    0.        2.7021
        0.7713    0.4399    0.4586    0.0355    0.        0.0015    4.0874
        0.6407    0.4841    0.5302    0.        0.2723    0.        2.7531
        0.9384    0.2396    0.2436    0.0008    0.0516    0.        2.6013
        0.7166    0.6576    0.2323    0.        0.007     0.0012    1.9698
        0.7802    0.4377    0.4425    0.0595    0.0198    0.0003    2.3369
        0.7671    0.3738    0.4918    0.        0.1734    0.0052    3.0098
        0.7659    0.4967    0.3769    0.        0.1572    0.        1.7791
        0.7448    0.3449    0.5504    0.        0.1526    0.        1.9758
        0.7696    0.3979    0.4991    0.0149    0.        0.        3.3411

        ...
\end{verbatim}}

{\noindent Second file created is the weight file esteveWeightOld.dat.wgt:}
{\scriptsize
\begin{verbatim}
        10.   // weight = 10
        10.   // weight = 10
        10.   // weight = 10
        10.   // weight = 10
        10.   // weight = 10
        10.   // weight = 10
        10.   // weight = 10
        10.   // weight = 10
        10.   // weight = 10
        10.   // weight = 10

        ...
\end{verbatim}}


%=======================================
\subsection{ Convert old partition file}
%=======================================
In the package directory MIXMOD/BIN/ there is an executable file named {\em convertOldPartitionFile}. This
program convert an old partition file to a new partition file (betwenn {\sc mixmod} 1.5 and 1.6). The number of
concerned samples (first number in old partition file) and the number of sample (first number on each line) are
removed. Moreover, samples which label are unknown, are replaced by a line vector of 0.\\
\noindent{\bf Synopsis}
{\scriptsize
\begin{verbatim}
       convertOldPartitionFile nameFile n k
\end{verbatim}}

\noindent{$n$ number of rows (number of samples)}\\
$k$: number of clusters\\
The conversion create a new partition file .lab ({\em nameFile.lab}) which contains a partition matrix of 0 and
1.\\
Consider the old partition file {\em geyserOldPartition.lab} in the directory {\em MIXMOD/EXAMPLES/DOC}:
{\scriptsize
\begin{verbatim}
        10
        1    0   1
        2    1   0
        3    0   1
        4    1   0
        5    0   1
        6    1   0
        7    0   1
        8    0   1
        9    1   0
        10   0   1
\end{verbatim}}

\noindent{Command for this example {\em MIXMOD/BIN/}:}
{\scriptsize
\begin{verbatim}
      convertOldPartitionFile ../../EXAMPLES/DOC/geyserOldPartition.lab 272 2
\end{verbatim}}

\noindent{Created file is {\em geyserOldPartition\_newPartition.lab}:}
{\scriptsize
\begin{verbatim}
        0   1    // sample 1 in component 2
        1   0
        0   1
        1   0
        0   1
        1   0
        0   1
        0   1
        1   0
        0   1    // sample 10 in component 2
        0   0    // sample 11 not affected

        ...

        0   0    // sample 272 not affected
\end{verbatim}}

\newpage
