This section outlines the operations and the available options of the
package for an on-line use. Instructions on the form of the input
and output files are also given. {\sc mixmod} is called by {\sc mixmod}'s executable file in the MIXMOD/BIN/ directory.\\
Results are stored in the output text files.\\

\noindent{\bf Synopsis}
{\scriptsize
\begin{verbatim}
       mixmod  input_filename (or mixmod.exe input_filename for windows)
\end{verbatim}}

\noindent{\bf Example}
{\scriptsize
\begin{verbatim}
       mixmod  ../EXAMPLES/geyser.xem
\end{verbatim}}

%======================
%======================
\subsection{Input file}
%======================
%======================

%===============================
\subsubsection{File structure for Cluster Analysis}
%===============================

{\sc mixmod} needs an input data file with arguments that reflect the chosen preferences or needs.\\
An input data file consists in a succession of keywords followed by values
that precise the caracteristics of the input data. They
make this input data file easier to write because there is no order in the definition except for
the strategies.

Two types of inputs are availables
\begin{itemize}
  \item Required Inputs

\begin{itemize}
  \item {\em NbLines} Specifies the number of rows in the data set file.
   \item {\em NbNbCluster} Specifies the length of the list of number of clusters.
  \item {\em ListNbCluster} Specifies the list of number of clusters defined by NbNbCluster.
  \item {\em DataFile} Specifies the path and name of the data file.
  \item {\em PbDimension} Specifies the problem dimension an integer which gives the number of colums in the data file.

 % \item In gaussian HD case
 % \begin{itemize}
 %   \item {\em SubDimensionFree} Specifies the intrinsic dimension of each class by a vector.
 %   \item {\em SubDimensionEqual} Specifies the intrinsic dimension of each class by an integer.
 % \end{itemize}
  \item In qualitative case
  \begin{itemize}
  \item{\em NbModality} Specifies the modalities on each variable.
  \end{itemize}

\end{itemize}
  \item Optional Inputs
 \begin{itemize}
 \item criterion,
 \item models,
% \item partition file,
 \item weight file,
 \item strategy.
 \end{itemize}
\end{itemize}

\noindent{Let's consider the input files {\it geyserDefault.xem}, which is in the package directory
MIXMOD/EXAMPLES/}. It contains the required inputs and all the optional inputs are initialized with default values
{\scriptsize
\begin{verbatim}
        NbLines
            272                      [Number of lines in the data file]
        PbDimension
            2                        [Dimension sample]
        NbNbCluster
            1                        [List of size 1]
        ListNbCluster
            2                        [2 components specified]
        DataFile
            ../DATA/geyser.dat       [Input data set (Old Faithful Geyser)]
\end{verbatim}}




To run this example under Linux/Unix, use the following command in MIXMOD/BIN/ directory.
{\scriptsize
\begin{verbatim}
        mixmod  ../EXAMPLES/geyserDefault.xem
\end{verbatim}}



\paragraph{Example of an input file with strategy geyserMiniUSER.xem }

{\scriptsize
\begin{verbatim}
        NbLines
            272
        PbDimension
            2
        NbNbCluster
            1
        ListNbCluster
            2
        DataFile
            ../DATA/geyser.dat

        NbStrategy
            1
        InitType
            USER                       [Starting parameter specified by user]
        InitFile
            ../DATA/geyser.init        [Input file for parameter initialisation for 2 clusters]
        NbAlgorithm
            1
        Algorithm
            EM
        StopRule
            NBITERATION
        StopRuleValue
            200
\end{verbatim}}


%\paragraph{Example of an input file with a known partition geyserMiniKnownLabel.xem}

%{\scriptsize
%\begin{verbatim}
%        NbLines
%            272
%        PbDimension
%            2
%        NbNbCluster
%            1
%        ListNbCluster
%            2
%        DataFile
%            ../DATA/geyser.dat
%
%        PartitionFile
%            ../DATA/geyser.part      [define partition file for known label]
%
%\end{verbatim}}





%\newpage





\paragraph{An input file with all options geyser.xem }


{\scriptsize
\begin{verbatim}
        NbLines
            272                      [Number of lines in the data file]
        PbDimension
            2                        [Dimension sample]
        NbCriterion
            1                        [One criterion]
        ListCriterion
            BIC                      [BIC criterion]
        NbNbCluster
            1                        [List of size 1]
        ListNbCluster
            2                        [2 components specified]
        NbModel
            1                        [One model]
        ListModel
            Gaussian_pk_Lk_I         [Gaussian spherical model [pk_Lk_I]]
        NbStrategy
            1                        [One strategy]
        InitType
            RANDOM                   [Starting parameter by random centers]
        NbAlgorithm
            1                        [One algorithm in the strategy]
        Algorithm
            EM                       [EM algorithm]
        StopRule
            NBITERATION              [Stopping rule for EM is number of iterations]
        StopRuleValue
            200                      [200 iterations desired]
        DataFile
            ../DATA/geyser.dat       [Input data set (Old Faithful Geyser)]
        WeightFile
            ../DATA/geyser.wgt       [Weight file for data]
\end{verbatim}}





{\noindent Many examples are available in directories MIXMOD/EXAMPLE and MIXMOD/TEST.}




%------------------------------------
\subsubsection{File structure for Discriminant Analysis}
%------------------------------------
To execute Discriminant Analysis with {\sc mixmod}, the user supplies
classified observations (training observations) and tests observations to be
classified.
Suppose that we have $2$ groups, and for each observation we know
the group membership. An estimation of the parameters is obtained
by maximizing the log-likelihood.

%{\scriptsize
%\begin{verbatim}
%        NbLines
%            272
%        PbDimension
%            2
%        NbCriterion
%            1
%        ListCriterion
%            CV
%        NbNbCluster
%            1
%        ListNbCluster
%            2
%        NbModel
%            1
%        ListModel
%            Gaussian_pk_Lk_Ck
%        NbStrategy
%            1                         [One strategy]
%        InitType
%            USER_PARTITION            [Initialisation by user partition]
%        InitFile
%            ../DATA/geyser.part       [Labels (input file) in directory MIXMOD/DATA/]
%        NbAlgorithm
%            1                         [One algorithm]
%        Algorithm
%            M                         [Maximization of log-Likelihood]
%        DataFile
%            ../DATA/geyser.dat
%        PartitionFile
%            ../DATA/geyser.part
%\end{verbatim}}


%{\scriptsize
%\begin{verbatim}
%        NbLines
%            200
%        PbDimension
%            15
%        SubDimensionEqual
%            4
%        NbCriterion
%            1
%        ListCriterion
%            CV
%        NbNbCluster
%            1
%        ListNbCluster
%            3
%        NbModel
%            1
%        ListModel
%            Gaussian_HDDA_pk_AkjBkQkD
%        NbStrategy
%            1                         [One strategy]
%        InitType
%            USER_PARTITION            [Initialisation by user partition]
%        InitFile
%            ../DATA/dataSimul_15D.part       [Labels (input file) in directory MIXMOD/DATA/]
%        NbAlgorithm
%            1                         [One algorithm]
%        Algorithm
%            M                         [Maximization of log-Likelihood]
%        DataFile
%            ../DATA/dataSimul_15D.dat
%        PartitionFile
%            ../DATA/dataSimul_15D.part
%\end{verbatim}}


{\scriptsize
 \begin{verbatim}
        NbLines
             1756
        PbDimension
             256
        NbCriterion
             1
        ListCriterion
             CV
        NbNbCluster
             1
        ListNbCluster
             3
        NbModel
             2
        ListModel
             Gaussian_HD_pk_AkjBkQkD
             subDimensionEqual
                10
             Gaussian_HD_pk_AkjBkQkDk
             subDimensionFree
                5
                6
                7
        NbStrategy
             1                         [One strategy]
        InitType
             USER_PARTITION            [Initialisation by user partition]
        InitFile
             ../DATA/USPS_358.part       [Labels (input file) in directory MIXMOD/DATA/]
        NbAlgorithm
             1                         [One algorithm]
        Algorithm
             M                         [Maximization of log-Likelihood]
        PartitionFile
             ../DATA/USPS_358.part
        DataFile
             ../DATA/USPS_358.dat
 \end{verbatim}
}


Running this file ({\it HD\_USPS\_358\_M.test} in directory MIXMOD/TEST) produces
estimated parameters (i.e. CVparameter.txt file, see output files (\ref{output})).\\

%Running this file (geyser.discriminant.training.xem in directory MIXMOD/EXAMPLES) produces
%estimated parameters (i.e. CVparameter.txt file, see output files (\ref{output})).\\

Now we use the estimated parameters to classify the optional test sample by the
MAP algorithm.

%{\scriptsize
%\begin{verbatim}
%        NbLines
%            5                                   [Remaining number of lines]
%        PbDimension
%            2                                   [Remaining data dimension]
%        NbNbCluster
%            1
%        ListNbCluster
%            2
%        NbModel
%            1
%        ListModel
%            Gaussian_pk_Lk_Ck
%        NbStrategy
%            1                                   [One strategy]
%        InitType
%            USER                                [Parameters specified by user]
%        InitFile
%            CVparameter.txt                    [Starting parameters are results of last step]
%        NbAlgorithm
%            1                                   [One algorithm]
%        Algorithm
%            MAP                                 [MAP algorithm]
%        DataFile
%            ../DATA/geyser.discriminant.dat     [Remaining sample to be classify]
%\end{verbatim}}


%{\scriptsize
%\begin{verbatim}
%        NbLines
%            3                                   [Remaining number of lines]
%        PbDimension                             [Remaining data dimension]
%            15
%        SubDimensionEqual
%            4
%        NbNbCluster
%            1
%        ListNbCluster
%            3
%        NbModel
%            1
%        ListModel
%            Gaussian_HDDA_pk_AkjBkQkD
%        NbStrategy
%            1                                   [One strategy]
%        InitType
%            USER                                [Parameters specified by user]
%        InitFile
%            CVparameter.txt                    [Starting parameters are results of last step]
%        NbAlgorithm
%            1                                   [One algorithm]
%        Algorithm
%            MAP                                 [MAP algorithm]
%        DataFile
%            ../DATA/dataSimul_15D_indiv.dat     [Remaining sample to be classify]
%\end{verbatim}}




{\scriptsize
  \begin{verbatim}
         NbLines
             17
        PbDimension
             256
        NbCriterion
             1
        ListCriterion
             BIC
        NbNbCluster
             1
        ListNbCluster
             3
        NbModel
             1
        ListModel
             Gaussian_HD_pk_AkjBkQkD
             subDimensionEqual
                10
        NbStrategy
             1
        InitType
             USER
        InitFile
             ../DATA/USPS_358.init
        NbAlgorithm
             1
        Algorithm
             MAP
        DataFile
             ../DATA/USPS_358.dat
  \end{verbatim}
}






Running this file ({\it HD\_USPS\_358\_MAP.test} in directory MIXMOD/TEST) produces
estimated classes/partition of the remaining data set (i.e. BIClabel.txt, BICpartition.txt files).

%-----------------------------
\subsubsection{Others options}
\label{output}
%-----------------------------
All global parameters, maximum numbers of iterations, number of random
centers initialisations, overflow, underflow,... are implemented and
can be modified in file XEMUtil.h in MIXMOD/SRC/ directory.

%=========================
%=========================
\subsection{Output files}
%=========================
%=========================
The Output files contain the results of the fit of mixture(s)
model(s) for each criterion type, they are created in the working directory.\\
This is an exhaustive list of the files
\begin{itemize}
 \item complete.txt,
 \item numericComplete.txt,
 \item errorMixmod.txt,
 \item errorModel.txt,
 \item $X$standard.txt,
 \item $X$numericStandard.txt,
 \item $X$partition.txt,
 \item $X$label.txt,
 \item $X$posteriorProbabilities.txt,
 \item $X$likelihood.txt,
 \item $X$numericLikelihood.txt,
 \item $X$parameter.txt
 \item $X$Error.txt,
 \item CVlabelClassification.txt,
 \item DCVinfo.txt,
 \item DCVnumericInfo.txt.
\end{itemize}

Here $X$ represents all available criteria BIC, ICL, NEC, CV, DCV.

%===========================
%\subsubsection{Complete output}
%===========================
\paragraph{Example of complete output}(result of {\it geyserMini2clusters.xem file}) in the directory MIXMOD/EXAMPLES

{\scriptsize
\begin{verbatim}
---------------------------------
     MIXMOD  Complete Output
---------------------------------
Number of samples 272.000000

        Strategy
        --------
        Initial start parameters method RANDOM
        Number of algorithms in the strategy 1
        Algorithm 1
          Type EM
          Stopping rule NBITERATION
          Number of iterations 100
          Set tolerance (xml criterion) 0.000100

                Number of Clusters 2
                ------------------

                        Model Type Gaussian Ellipsoidal Model pk_Lk_C
                        ----------

                        BIC Criterion Value 2322.971927
                        Component 1
                        ---------
                        Mixing proportion 0.642879
                        Mean 4.292206 79.996273
                        Covariance matrix
                                0.145346 0.830099
                                0.830099 40.903602

                        Component 2
                        ---------
                        Mixing proportion 0.357121
                        Mean 2.039681 54.516886
                        Covariance matrix
                                0.098361 0.561759
                                0.561759 27.680967


        Strategy
        --------
        Initial start parameters method RANDOM
        Number of algorithms in the strategy 1
        Algorithm 1
          Type EM
          Stopping rule NBITERATION
          Number of iterations 100
          Set tolerance (xml criterion) 0.000100

                Number of Clusters 3
                ------------------

                        Model Type Gaussian Ellipsoidal Model pk_Lk_C
                        ----------

                       BIC Criterion Value 2327.088167
                        Component 1
                        ---------
                        Mixing proportion 0.642748
                        Mean 4.292448 79.998690
                        Covariance matrix
                                0.158302 0.756685
                                0.756685 36.185614

                        Component 2
                        ---------
                        Mixing proportion 0.140194
                        Mean 1.974277 49.482145
                        Covariance matrix
                                0.042067 0.201082
                                0.201082 9.615981

                        Component 3
                        ---------
                        Mixing proportion 0.217058
                        Mean 2.082576 57.777045
                        Covariance matrix
                                0.097293 0.465063
                                0.465063 22.239891
\end{verbatim}}
