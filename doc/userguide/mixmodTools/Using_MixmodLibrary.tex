{\sc mixmod} can be used in another program as a library in the conditions of ({\it see
www.gnu.org/copyleft/gpl.html)} GNU GPL license.
We will quickly describe how to
use Mixmod as a library in this section but the reference guide is the {\bf Software Documentation}.\\


%------------------------------------------
\subsection{How to build a Mixmod project ?}
%------------------------------------------
To build a such project, the user must link his program to
\begin{itemize}
	\item {\sc mixmod} library (in \textit{LIB/MIXMOD} directory)
	\item {\sc newmat} library (in \textit{LIB/NEWMAT} directory)
\end{itemize}
The \textit{MIXMOD} directory is a example of a such program \textit{MIXMOD/SRC} directory contains \textit{main.cpp} which have to be linked to {\sc mixmod} and {\sc newmat} libraries.
To build this project, the user can use'cmake' tools (but he is free to use another way !) by taping the following lines in \it{MIXMOD} directory
\begin{itemize}
  \item camke . (note the '.')
  \item make
  \item make install
\end{itemize}
So,
\begin{itemize}
  \item libnewmat,
  \item libmixmod,
  \item mixmod executable
\end{itemize}
will be created.\\
So, to build a software using the {\sc mixmod} library, the user has to replace \textit{main.cpp} by another \textit{main} which does not use input and output files. Such \textit{main}s are available in the \textit{MIXMOD/EXAMPLES} directory.\\


%------------------------------------------------
\subsection{How to create a {\it main.cpp} ?}
%------------------------------------------------
This file must
\begin{itemize}
  \item create an {\it XEMInput} object which contains all input informations,
  \item create an {\it XEMMain} object using the {\it XEMInput} object,
  \item create an {\it XEMOutput} object which contains all output informations,
  \item call the {\it run} method of the {\it XEMMain} object with the {\it XEMOutput} object.
\end{itemize}




%--------------------
\subsection{Examples}
%--------------------

%--------------------------------------------
\subsubsection{First example minimal main.cpp}
%--------------------------------------------
This example is the simplest way to create input, output and xmain objects.\\
See MIXMOD/EXAMPLES/main\_mini.cpp.\\
{\scriptsize
\begin{verbatim}
  #include "../LIB/MIXMOD/XEMMain.h"
  #include "../LIB/MIXMOD/XEMGaussianData.h"

  int main(int argc, char * argv[]){

  XEMOutput * output = NULL;
  try{
    int nbSample = 272;
    int pbDimension = 2;

    int nbNbCluster = 1;
    int tabNbCluster[1];
    tabNbCluster[0]=2;

    char dataFileName[100];
    strcpy(dataFileName,"../DATA/geyser.dat");
    XEMData * data = new XEMGaussianData(nbSample, pbDimension, dataFileName);

    XEMInput * input = new XEMInput(nbSample, pbDimension, nbNbCluster, tabNbCluster, data);
    input->finalize();

    //XEMMain
    XEMMain xmain(input);

    //Create a XEMOutput object
    //-------------------------
    output = new XEMOutput(input, xmain);

    // xmain run
    xmain.run(output);

    // output printing ...
    cout<<"model paramter "<<endl;
    cout<<"--------------"<<endl;
    XEMModelOutput * modelOutput = output->_tabBestModelOutput[0];
    XEMParameter * param = modelOutput->_param;
    param->edit();

    // delete
    delete data;
    delete input;
  }

  catch (XEMErrorType errorType){
    XEMError error(errorType);
    error.run();
  }

  delete output;

  return 0;
}
\end{verbatim}}

So, the user can use the output object to edit informations or to use them in his own objects
(see {\bf Software Documentation} to know more about the {\it XEMOutput} class).\\


%--------------------------------------------------------------
\subsubsection{How to give specific information to an input object ?\\}
%--------------------------------------------------------------
Several methods of {\it XEMInput} class can be called to change default values (see {\bf Software Documentation}).\\
\\
This other example explain how to specify model types in the input object.\\
See MIXMOD/EXAMPLES/main\_model.cpp.
{\scriptsize
\begin{verbatim}
  #include "../LIB/MIXMOD/XEMMain.h"
  #include "../LIB/MIXMOD/XEMModelType.h"
  #include "../LIB/MIXMOD/XEMGaussianData.h"

  int main(int argc, char * argv[]){

    XEMOutput * output = NULL;

    try{
      int nbSample = 272;
      int pbDimension = 2;

      int nbNbCluster = 1;
      int tabNbCluster[1];
      tabNbCluster[0]=2;

      char dataFileName[100];
      strcpy(dataFileName,"../DATA/geyser.dat");
      XEMData * data = new XEMGaussianData(nbSample, pbDimension, dataFileName);

      XEMInput * input = new XEMInput(nbSample, pbDimension, nbNbCluster, tabNbCluster, data);

      XEMModelType * tabModelType[28];
      tabModelType[0] = new XEMModelType(Gaussian_p_L_I);
      tabModelType[1] = new XEMModelType(Gaussian_p_Lk_I);
      tabModelType[2] = new XEMModelType(Gaussian_p_L_B);
      tabModelType[3] = new XEMModelType(Gaussian_p_Lk_B);
      tabModelType[4] = new XEMModelType(Gaussian_p_L_Bk);
      tabModelType[5] = new XEMModelType(Gaussian_p_Lk_Bk);
      tabModelType[6] = new XEMModelType(Gaussian_p_L_C);
      tabModelType[7] = new XEMModelType(Gaussian_p_Lk_C);
      tabModelType[8] = new XEMModelType(Gaussian_p_L_D_Ak_D);
      tabModelType[9] = new XEMModelType(Gaussian_p_Lk_D_Ak_D);
      tabModelType[10] = new XEMModelType(Gaussian_p_L_Dk_A_Dk);
      tabModelType[11] = new XEMModelType(Gaussian_p_Lk_Dk_A_Dk);
      tabModelType[12] = new XEMModelType(Gaussian_p_L_Ck);
      tabModelType[13] = new XEMModelType(Gaussian_p_Lk_Ck);
      tabModelType[14] = new XEMModelType(Gaussian_pk_L_I);
      tabModelType[15] = new XEMModelType(Gaussian_pk_Lk_I);
      tabModelType[16] = new XEMModelType(Gaussian_pk_L_B);
      tabModelType[17] = new XEMModelType(Gaussian_pk_Lk_B);
      tabModelType[18] = new XEMModelType(Gaussian_pk_L_Bk);
      tabModelType[19] = new XEMModelType(Gaussian_pk_Lk_Bk);
      tabModelType[20] = new XEMModelType(Gaussian_pk_L_C);
      tabModelType[21] = new XEMModelType(Gaussian_pk_Lk_C);
      tabModelType[22] = new XEMModelType(Gaussian_pk_L_D_Ak_D);
      tabModelType[23] = new XEMModelType(Gaussian_pk_Lk_D_Ak_D);
      tabModelType[24] = new XEMModelType(Gaussian_pk_L_Dk_A_Dk);
      tabModelType[25] = new XEMModelType(Gaussian_pk_Lk_Dk_A_Dk);
      tabModelType[26] = new XEMModelType(Gaussian_pk_L_Ck);
      tabModelType[27] = new XEMModelType(Gaussian_pk_Lk_Ck);

      input->setModelType(28, tabModelType);
      input->finalize();

      //XEMMain
      XEMMain xmain(input);

      //Create a XEMOutput object
      //-------------------------
      output = new XEMOutput(input, xmain);

      // xmain run
      xmain.run(output);

      // output printing ...
      cout<<"best model paramter "<<endl;
      cout<<"-------------------"<<endl;
      XEMModelOutput * modelOutput = output->_tabBestModelOutput[0];
      XEMParameter * param = modelOutput->_param;
      param->edit();

      // delete
      delete data;
      delete input;
      for (int i=0; i<28; i++){
        delete tabModelType[i];
      }
    }

    catch (XEMErrorType errorType){
      XEMError error(errorType);
      error.run();
    }

  delete output;

  return 0;
}
\end{verbatim}}


%----------------------------
\subsubsection{Other examples\\}
%----------------------------
The MIXMOD/EXAMPLES directory contains some {\it main} examples
\begin{itemize}
	\item main\_criterion.cpp to change default value of the criterion,
	\item main\_strategy1.cpp to set the default strategy to the input object,
\item main\_strategy2.cpp to set a new strategy to the input object {\it USER} initialization and 200 iterations of {\it SEM} algorithm,
	\item main\_discriminantAnalysis.cpp an example for discriminant analysis in qualitative case.
\end{itemize}


%\newpage
